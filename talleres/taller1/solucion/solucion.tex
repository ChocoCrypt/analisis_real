% article example for classicthesis.sty
\documentclass[10pt,a4paper]{article} % KOMA-Script article scrartcl
\usepackage{import}
\usepackage{xifthen}
\usepackage{pdfpages}
\usepackage{transparent}
\newcommand{\incfig}[1]{%
    \def\svgwidth{\columnwidth}
    \import{./figures/}{#1.pdf_tex}
}
\usepackage{lipsum}     %lorem ipsum text
\usepackage{titlesec}   %Section settings
\usepackage{titling}    %Title settings
\usepackage[margin=10em]{geometry}  %Adjusting margins
\usepackage{setspace}
\usepackage{listings}
\usepackage{amsmath}    %Display equations options
\usepackage{amssymb}    %More symbols
\usepackage{xcolor}     %Color settings
\usepackage{pagecolor}
\usepackage{mdframed}
\usepackage[spanish]{babel}
\usepackage[utf8]{inputenc}
\usepackage{longtable}
\usepackage{multicol}
\usepackage{graphicx}
\graphicspath{ {./Images/} }
\setlength{\columnsep}{1cm}

% ====| color de la pagina y del fondo |==== %



\begin{document}
    %========================{TITLE}====================%
    \title{{  Primer Taller Análisis Real  }}
    \author{{Rodrigo Castillo}}
    \date{\today}

    \maketitle


     % ====| Loguito |==== %
    %=======================NOTES GOES HERE===================%
    \section{La suma o el producto de dos numeros irracionales es irracional?}

    \section{si $x$ es racional , $x \not= 0$ , $y \in I$ , demostrar que $x+y
    , x-y , xy , x/y , y/x \in I $ }

    \section{Muestre que la intersección arbitraria de conjuntos inductivos es
    un conjunto inductivo}

    \section{su x es un numero real arbitrario , demostrar que existe un único
    entero $n$ tal que $n<x<x+1$ Este número $n$ se denomina la parte entera de
    $x$ }

    \section{muestre que no existe un numero nacional $r$ tal que $r ^{2} = 3$}

    \section{muestre que $\sqrt{2} + \sqrt{3} \in I$}























    %=======================NOTES ENDS HERE===================%

    % bib stuff
    \nocite{*}
    \addtocontents{toc}{{}}
    \addcontentsline{toc}{section}{\refname}
    \bibliographystyle{plain}
    \bibliography{../Bibliography}
\end{document}
